Artificial Intelligence (AI) systems have become increasingly integral to modern cloud-based applications, powering services in healthcare, finance, logistics, and countless other domains \cite{zhang2021survey,ghosh2021cloud}. These AI workloads often rely on scalable and distributed cloud infrastructures that provide the computational power, storage, and elasticity required to train, deploy, and maintain sophisticated models. However, ensuring the \emph{availability} of such AI systems in cloud environments remains a significant challenge.

High availability (HA) refers to the ability of a system to remain operational and accessible with minimal downtime, even in the face of faults or failures \cite{tanenbaum2017distributed,aws2023ha}. While HA is a well-established goal in general cloud infrastructure, its application to AI workloads introduces additional complexities. AI services are typically resource-intensive, often requiring specialized hardware such as GPUs, and are sensitive to disruptions due to their stateful nature and dynamic model updates \cite{lee2022resilient,ghosh2021cloud}. This raises critical questions about how traditional HA strategies apply to AI and whether they require adaptation or innovation.

Despite the increasing importance of making AI systems robust and continuously available, there is a noticeable lack of consolidated knowledge on how HA is being implemented in practice within AI deployments on the cloud \cite{verma2023resilience,zhang2021survey}. While various techniques exist—such as replication, failover, and distributed orchestration—it remains unclear how widely these are adopted, which AI subfields they target, and what gaps or challenges persist in the literature \cite{bernstein2014containers,tanenbaum2017distributed}.

To address this, we conducted a \emph{Systematic Mapping Study} (SMS) to investigate how high availability strategies are applied to support AI systems in cloud computing environments. The objective is not to propose new technical solutions, but rather to map existing research, identify common approaches, understand the types of AI workloads considered, and expose gaps in current knowledge. Our mapping covers peer-reviewed literature from 2015 to 2025 and provides a structured overview of the state of research in this interdisciplinary area.

The remainder of this document is organized as follows. Section~\ref{sec:background} provides foundational concepts related to cloud computing, artificial intelligence, and high availability. Section~\ref{sec:planning} outlines the methodology used for our systematic mapping study, including the research protocol, databases, and inclusion/exclusion criteria. Section~\ref{sec:results} presents an analysis of the selected studies, summarizes key findings, and discusses observed research gaps. Finally, we conclude with reflections and directions for future work.
